\documentclass[journal=apchd5,manuscript=article,layout=twocolumn]{achemso}

\setkeys{acs}{articletitle=true}
\setkeys{acs}{keywords = true}

% apchd5: ACS Phot
% nalefd: Nano Lett
% acsodf: ACS Omega
% jpclcd: JPCL
% jpccck: JPCC

\usepackage[utf8]{inputenc}
\usepackage[T1]{fontenc}

\usepackage[]{amsmath}
\usepackage[]{lipsum}

\usepackage[version=3]{mhchem} % Formula subscripts using \ce{}

\DeclareGraphicsExtensions{%
    .pdf,.PDF,%
    .png,.PNG,%
    .jpg,.jpeg}


\begin{document}

\begin{abstract}
  \lipsum[1-1]
\end{abstract}


\section{Introduction}

\lipsum[2-4]







\begin{acknowledgement}

The authors thank Mats Dahlgren for version one of \textsf{achemso},
and Donald Arseneau for the code taken from \textsf{cite} to move
citations after punctuation.

\end{acknowledgement}


\begin{suppinfo}
This will usually read something like: ``Experimental procedures and
characterization data for all new compounds. The class will
automatically add a sentence pointing to the information on-line:
\end{suppinfo}


\bibliography{achemso-demo}

\begin{tocentry}
Some journals require a graphical entry for the Table of Contents.
This should be laid out ``print ready'' so that the sizing of the
text is correct.

Inside the \texttt{tocentry} environment, the font used is Helvetica
8\,pt, as required by \emph{Journal of the American Chemical
Society}.

The surrounding frame is 9\,cm by 3.5\,cm, which is the maximum
permitted for  \emph{Journal of the American Chemical Society}
graphical table of content entries. The box will not resize if the
content is too big: instead it will overflow the edge of the box.

This box and the associated title will always be printed on a
separate page at the end of the document.
\end{tocentry}

\end{document}